\documentclass{exam}

\usepackage[utf8]{inputenc}
\usepackage[brazil]{babel}
\usepackage{amsmath}
\usepackage{fancyvrb}

\title{
	\normalsize INF2604 - Geometria Computacional\\
	\LARGE \textbf{Exercício 1: Conceitos Básicos}
	}
	
\author{Pedro Torres}

\date{Agosto de 2019}

\begin{document}

\maketitle

\begin{questions}

\question

\begin{Verbatim}[tabsize=4]
	pseudo-angulo(x, y):
		se y >= 0:
			se x >= 0:
				se x >= y:
					retorne y/x
				se não:
					retorne 2 - x/y
			se -x <= y:
				retorne 2 + (-x)/y
			se não:
				retorne 4 - y/(-x)
		se x < 0:
			se -x >= -y:
				retorne 4 + (-y)/(-x)	
			se não:
				retorne 6 - (-x)/(-y)
		se x <= -y
			retorne 6 + x/(-y)
		se não:
			retorne 8 - (-y)/x

\end{Verbatim}

\question Seja $\overrightarrow{v_1} = (x_1, y_1), \overrightarrow{v_2} = (x_2, y_2), \overrightarrow{v_3} = (x_3, y_3)$. Queremos determinar o ponto de singularidade, ou seja, $\overrightarrow{v} = 0$. Para isto, resolvemos o seguinte sistema:

\begin{equation}
	\begin{cases}
		\lambda_1 x_1 + \lambda_2 x_2 + \lambda_3 x_3 = 0 \\
		\lambda_1 y_1 + \lambda_2 y_2 + \lambda_3 y_3 = 0 \\
		\lambda_1 + \lambda_2 + \lambda_3 = 1	
	\end{cases}
\end{equation}

\question
\begin{parts}

\part 2-manifold, possui fronteira.
\part 1-manifold, não possui fronteira.
\part 3-manifold, possui fronteira.
\part 2-manifold, não possui fronteira.
\end{parts}

\end{questions}

\end{document}